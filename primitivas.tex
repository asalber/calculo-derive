%$HeadURL: https://practicas-derive.googlecode.com/svn/trunk/primitivas.tex $
%$LastChangedDate: 2009-11-16 16:11:46 +0100 (lun, 16 nov 2009) $
%$LastChangedRevision: 5 $
%$LastChangedBy: asalber $

\chapter{Cálculo de primitivas}

\section{Fundamentos teóricos}
El cálculo de primitivas se puede considerar como un proceso inverso
al cálculo de derivadas, y por eso también se suele llamar
antiderivada a la primitiva de una función. Desgraciadamente, y a
diferencia del cálculo de derivadas, no existe un procedimiento
infalible que permita calcular la primitiva de una función siempre
que exista. En esta práctica se presentan las técnicas más
habituales para el cálculo de primitivas, las cuales, con un poco de
pericia, permiten calcular gran parte de las primitivas de las
funciones más habituales.

\subsection*{Primitivas e Integrales}
\subsubsection*{Función Primitiva}

Se dice que la función $F(X)$ es una \emph{función primitiva} de
$f(x)$ si se verifica que $F'(x)=f(x)$ $\forall x \in \dom f$.

Como dos funciones que difieran en una constante tienen la misma
derivada, si $F(x)$ es una función primitiva de $f(x)$ también lo será toda función de la forma $F(x)+k$ $\forall k \in \mathbb{R}$.\\


\subsubsection*{Función integral indefinida}

Se llama \emph{función integral indefinida} de la función $f(x)$ al
conjunto de todas sus funciones primitivas y se representa como:

\[
\ \int{f(x)}\,dx=F(x)+C
\]
siendo $F(x)$ una función primitiva de $f(x)$ y $C$ una constante arbitraria.\\


\subsubsection*{Linealidad de la integral}

Dadas dos funciones $f(x)$ y $g(x)$ que admiten primitiva, y una
constante $k \in \mathbb{R}$ se verifica que:

\[
\ \int{(f(x)+g(x))}\,dx=\int{f(x)}\,dx+\int{g(x)}\,dx
\]
y:
\[
\ \int{kf(x)}\,dx=k\int{f(x)}\,dx
\]\\


\subsection*{Técnicas generales de integración}

\subsubsection*{Integración por partes}

Dadas $f$ y $g$, dos funciones derivables de $x$, se verifica:

\[
\ \int{f'(x)g(x)}\,dx=f(x)g(x)-\int{g'(x)f(x)}\,dx\overline{}
\]
o con notación diferencial, si $u$ y $v$ son funciones derivables de
$x$:

\[
\ \int{u}\,dv=uv-\int{v}\,du
\]


Al emplear el método de integración por partes se debe realizar la
elección de $u$ y $dv$ de tal forma que las integrales que haya que
realizar sean lo más sencillas posibles. Así, si se tiene que
calcular $\int{x \sen x}\,dx$ se deberá elegir $u=x$ y $dv=\sen x\,
dx$, con lo que $du=dx$ y $v=-\cos x$, resultando:

\[
\ \int{x \sen x}\,dx=-x\cos x+\sen x
\]

\subsubsection*{Fórmulas de reducción}

Las fórmulas de reducción permiten simplificar el cálculo cuando hay
que aplicar la integración por partes varias veces consecutivas. Si
se tiene que calcular una integral indefinida $I_{n}$ que depende de
un número natural $n$, las fórmulas de reducción nos permitirán
expresar $I_{n}$ en función de $I_{n-1}$, es decir se obtendrá una
relación recurrente del tipo:

\[
\ I_{n}=f(I_{n-1},x,n)
\]
con lo que calculando una integral se pueden obtener fácilmente las
demás.

Si se desea calcular $I_{n}=\int{x^{n}e^{x}}\,dx$, se debe elegir
$u=x^{n}$ y $dv=e^{x}\,dx$, con lo que $du=nx^{n-1}\,dx$ y
$v=e^{x}$, obteniéndose:

\[
\ I_{n}=\int{x^{n}e^{x}}\,dx=x^{n}e^{x}-n\int{x^{n-1}e^{x}}\,dx=x^{n}e^{x}-nI_{n-1}
\]

\subsubsection*{Cambio de variable}

Al calcular una integral de la forma $\int f(g(x))g'(x)\,dx$, donde
$g$ es una función con derivada continua y $f$ es una función
continua, haciendo el cambio de variable $u=g(x)$ se obtiene:

\[
\ \int{f(g(x))g'(x)}dx=\int{f(u)\,du}
\]
y una vez calculada esta integral se deshace el cambio de variable
efectuado para obtener la integral deseada.

Así, si se desea calcular:


\[
\ \int{\frac{dx}{x\log x}}
\]
puede hacerse el cambio de variable $u=\log x$ con lo que:

\[
\ \int{\frac{dx}{x\log x}}=\int{\frac{du}{u}}=\log |u|+ C= \log |\log x| + C
\]

\subsubsection*{Integración de funciones racionales}

Toda función racional se puede escribir como suma de un polinomio (que tiene primitiva inmediata) más una función racional propia, es decir, una función racional en la que el grado del numerador sea menor que el grado del denominador. A su vez, toda función racional propia puede expresarse como suma de fracciones simples de los tipos siguientes:
\[
\begin{array}{cl}
\dfrac{A}{(x-a)}& \textrm{: Correspondiente a raíces reales simples del denominador.}\\
\dfrac{A}{(x-a)^{n}}& \textrm{: Correspondiente a raíces reales múltiples del denominador.}\\
\dfrac{Ax+B}{(x-r)^{2}+s^{2}}& \textrm{: Correspondiente a raíces complejas simples del denominador.}\\
\dfrac{Ax+B}{((x-r)^{2}+s^{2})^{n}} & \textrm{: Correspondiente a raíces complejas múltiples del denominador.}
\end{array}
\]
con $n>1$, por lo que usando la linealidad de la integral, basta saber calcular una primitiva de cada una de estas fracciones simples para que se pueda calcular la primitiva de cualquier función racional.

Las primitivas de las dos primeras fracciones simples son inmediatas
ya que:
\[
\int \frac{A}{x-a}\,dx=A\log|x-a|+C
\quad \textrm{y} \quad
\int \frac{A}{(x-a)^n}\,dx=\frac{-A}{(n-1)(x-a)^{n-1}}+C \textrm{ si $n\neq 1$}.
\]

En cuanto a las dos últimas fracciones simples, con ligeras
transformaciones, siempre se obtienen primitivas que combinan
logaritmos con arcotangentes.

\newpage

\section{Ejercicios resueltos}

\begin{indicacion}
{Para obtener primitivas de funciones se puede utilizar el menú
\menu{Cálculo->Integrales}, en el que se selecciona
\opcion{Indefinida} y se introduce el valor de la constante de
integración deseado. En lugar de utilizar el menú indicado se puede
pinchar en el botón \boton{Integrales}.

También se pueden obtener primitivas de funciones introduciendo en
la línea de edición \comando{INT(u,x)}, que nos da la primitiva de
la función $u$ respecto de la variable $x$ sin incluir constante de
integración, o bien \comando{INT(u,x,c)}, que incluye dicha
constante.

Aunque la mayor parte de las primitivas se pueden obtener con los
procedimientos indicados, en algunas ocasiones puede ser necesario
emplear procedimientos especiales de integración, como pueden ser la
integración por partes o por sustitución. Para ello se pueden
emplear las funciones:\comando{INT\_PARTS(u,v,x)}, que emplea la
integración por partes para encontrar la primitiva de $uv$ respecto
a $x$; y \comando{INT\_SUBST(y,x,u)}, que obtiene la primitiva de la
función $y(x)$ haciendo el cambio de variable $u=f(x)$, calculando
la integral de $y(u)$ y sustituyendo $u$ por $f(x)$ en el resultado.

}
\end{indicacion}

\begin{enumerate}[leftmargin=*]
\item Calcular las siguientes integrales:

\begin{enumerate}
\item $ \int{x^{2} \log{x}\,dx}$

\begin{indicacion}
{Introducir $x^{2}\log(x)$ en la línea de edición. Utilizar el menú
\menu{Cálculo->Integrales}, eligiendo \opcion{Integral Indefinida},
\opcion{Constante} $c$ y pinchar en el botón \boton{Simplificar}. }
\end{indicacion}

\item $ \int{\dfrac{5x^{2}+4x+1}{x^{5}-2x^{4}+2x^{3}-2x^{2}+x}\,dx}$
\begin{indicacion}
{Proceder de la forma indicada en el apartado \emph{a}). }
\end{indicacion}

\item $ \int{\dfrac{6x+5}{(x^{2}+x+1)^{2}}\,dx}$
\begin{indicacion}
{Proceder de la forma indicada en el apartado \emph{a}). }
\end{indicacion}
\end{enumerate}

\item Calcular la primitiva de la función $f(x)=x^{2}\log x$ mediante integración por partes.
\begin{indicacion}
{Introducir sucesivamente en la línea de edición $u:=x^{2}$ y
$v:=\log(x)$, para definir $u$ y $v$ respectivamente y
\comando{INT\_PARTS(u,v,x)}, para obtener mediante integración por
partes la primitiva de $u \cdot v$ respecto a $x$, y aplicar el
símbolo $=$ para obtener el resultado. Una vez hecho esto, se
introduce sucesivamente en la línea de edición $u:=$ y $v:=$ para
anular las definiciones de $u$ y $v$. También podía haberse hecho el
cálculo deseado sin necesidad de definir $u$ y $v$, introduciendo
\comando{INT\_PARTS}$(x^2,\log(x),x)$. }
\end{indicacion}

\item Calcular la primitiva de la función $g(x)=x\sen x^2$ mediante sustitución.
\begin{indicacion}
{ Introducir sucesivamente en la línea de edición $y:=x\sin(x^{2})$
y $u:=x^{2}$, para definir $y$ y $u$ respectivamente y
\comando{INT\_SUBST(y,x,u)}, para obtener mediante integración por
sustitución la primitiva de $y(x)$ respecto a $x$ haciendo el cambio
$u=f(x)$, y aplicar el símbolo $=$ para obtener el resultado. Una
vez hecho esto, se introduce sucesivamente en la línea de edición
$y:=$ y $u:=$ para anular las definiciones de $y$ y $u$. También
podía haberse hecho el cálculo deseado sin necesidad de definir $y$
y $u$, introduciendo \comando{INT\_SUBST}$(x \sin(x^{2}),x,x^2)$. }
\end{indicacion}
\end{enumerate}

\section{Ejercicios propuestos}
\begin{enumerate}[leftmargin=*]
\item Calcular las siguientes integrales:
\begin{enumerate}
\item $ \int{\dfrac{2x^{3}+2x^{2}+16}{x(x^{2}+4)^{2}}\,dx}$
\item $ \int{\dfrac{1}{x^{2}\sqrt{4+x^{2}}}\,dx}$
\end{enumerate}
\item Calcular la primitiva de $f(x)=3^{x}\cos{x}$ mediante integración por partes.
\item Calcular la primitiva de $g(x)=\sen({\log{x}})$ mediante integración por sustitución.
\item Obtener una fórmula de reducción para la integral $I_{n}=\int{x^{n}e^{x}\,dx}$.
\end{enumerate}