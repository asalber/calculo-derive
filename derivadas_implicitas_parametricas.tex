% Author: Alfredo Sánchez Alberca (asalber@ceu.es)
\chapter{Derivadas implícitas y paramétricas}

\section{Fundamentos teóricos}
La forma habitual de representar una función real de variable real
es mediante la \emph{representación explícita} $y=f(x)$, donde la
variable $y$ aparece despejada a la izquierda de la igualdad, y a la
derecha aparece una expresión que sólo depende de la variable $x$, y
que expresa la dependencia de $y$ con respecto a $x$. No obstante,
esta no es la única forma de representar una función, ni todas las
funciones pueden representarse de forma explícita. Además de las
representación gráficas, las otras representaciones más habituales
de funciones son las representaciones \emph{implícitas} y
\emph{paramétricas}. En esta práctica se introducen estas
representaciones y se muestra cómo calcular derivadas de funciones
con estas representaciones.

\subsection*{Funciones implícitas}

Dada una función $F$ de las variables $x$ e $y$, se dice que la ecuación $F(x,y)=0$ expresa de una manera implícita la relación que existe entre dichas variables. Así, se puede decir que $F(x,y)=0$ define implícitamente la dependencia de $y$ con respecto a $x$ y la de $x$ con respecto a $y$. Pero no siempre se puede expresar explícitamente $y$ como función de $x$, de la forma $y=f(x)$, ni siquiera a nivel local, por lo que es importante saber hacer el cálculo de la derivada de $y$ con respecto a $x$ sin necesidad de despejarla.

Para ello se deriva $F$ respecto a $x$, teniendo en cuenta que, como $y$ es función de $x$, cuando derivemos algún término que contenga la $y$ tendremos que aplicar la regla de la cadena, escribiendo $y'$ para designar $\dfrac{dy}{dx}$.

Así, por ejemplo, supongamos que se quiere calcular $y'$, sabiendo que $x$ e $y$ están relacionadas por la ecuación $x^2-2y^3+4y-2=0$. Es claro que a partir de la ecuación dada resultará difícil expresar $y$ como función explícita de $x$, por lo que tendremos que realizar la derivación respecto a $x$ de la propia ecuación, con lo que se obtiene:

\[
\ 2x-6y^{2}y'+4y'=0
\]
y, despejando $y'$ queda:

\[
\ y'=\frac{x}{3y^{2}-2}
\]

\subsection*{Funciones paramétricas}

Se considera una función $y=f(x)$, que se puede expresar en un
entorno de $(x_{0},y_{0})$ siendo $y_{0}=f(x_{0})$, mediante la
expresión de $x$ e $y$ como funciones de un parámetro $t$:
\[
\left\{
\begin{array}{l}
x=x(t)\\
y=y(t)
\end{array}
\right.
\]
con $x_{0}=x(t_{0})$.

Si $x(t)$ es derivable en $t_{0}$ y $f(x)$ es derivable en $x_{0}$,
aplicando la regla de la cadena se verifica que:
\[
\ y'(t_{0})=f'(x_{0})x'(t_{0})
\]
con lo que, si $x'(t_{0})\neq0$, obtenemos:
\[
\ f'(x_{0})=\frac{y'(t_{0})}{x'(t_{0})}.
\]

Por ejemplo, supongamos que tenemos la siguiente función
paramétrica:
\[
\left\{
\begin{array}{l}
x(t)=t^{2}+3\\
y(t)=\log t
\end{array}
\right.
\]
y se desea calcular la derivada de $y$ respecto a $x$ para $t_{0}=1$.

El valor de $x$ que corresponde al valor $t_{0}=1$ es
$x_{0}=x(1)=4$. Derivando $x$ e $y$ respecto a $t$ se obtiene:

\begin{align*}
x'(t) &=2t\\
y'(t) &=\frac{1}{t}
\end{align*}
con lo que:
\begin{align*}
x'(t_{0})&=x'(1)=2\\
y'(t_{0})&=y'(1)=1
\end{align*}
y por consiguiente:
\[
\frac{dy}{dx}(t_0)=f'(x_{0})=f'(4)=\frac{y'(t_{0})}{x'(t_{0})}=\frac{y'(1)}{x'(1)}=\frac{1}{2}.
\]

\newpage

\section{Ejercicios resueltos}
\begin{indicacion}
Para calcular Derivadas de Funciones Implícitas o Paramétricas es
preciso:
\begin{enumerate}
\item Abrir previamente el archivo \variable{DifferentiationApplications} Para ello se ejecuta el menú \menu{Archivo > Abrir} y aparece un
cuadro que en el campo \opcion{Buscar en} tiene \opcion{Users}. Se pulsa el botón \boton{Subir un nivel}, se selecciona \opcion{Math} y se
abre el archivo \variable{DifferentiationApplications}.

\item Una vez hecho esto, si se desea realizar una derivación implícita se empleará la función \comando{IMP\_DIF(u,x,y,n)} para calcular la
derivada de orden $n$ de $y$ con respecto a $x$, siendo $u(x,y)=0$ la ecuación que expresa la relación entre $x$ e $y$ de manera implícita;
y si se desea realizar una derivación en paramétricas, se utilizará la función \comando{PARA\_DIF(v,t,n)} para calcular la derivada de orden
$n$ de $y$ con respecto a $x$ en términos del parámetro $t$, siendo $v$ un vector de la forma $v=[x(t),y(t)]$, donde $x(t)$ e $y(t)$ son
expresiones que dependen del parámetro $t$.
\end{enumerate} 
\end{indicacion}

\begin{enumerate}[leftmargin=*]
\item Calcular la derivada de $y$ con respecto a $x$, sabiendo que $x^{2}+y^{2}=1$.
\begin{indicacion}
Según la indicación anterior, introducir sucesivamente en la línea de edición, $u:=x^{2}+y^{2}-1$ para definir la variable $u$ como
$x^2+y^2-1$, y \comando{IMP\_DIF(u,x,y,1)} para calcular la derivada de orden 1 de $y$ con respecto a $x$, y aplicar el símbolo \boton{=}
para obtener el resultado. Una vez hecho esto, se introduce en la línea de edición $u:= $ para anular la definición de $u$.
\end{indicacion}

\item Calcular la derivada de $y$ con respecto a $x$, sabiendo que $x^{3}-3xy^{2}+y^{3}=1$.
\begin{indicacion}
Hacer lo mismo que en el ejercicio $1$ definiendo $u$ como $x^3-3xy^2+y^3-1$.
\end{indicacion}


\item Calcular la ecuación de la tangente a la curva $x^{3}-3xy^{2}+y^{3}=1$ en el punto de abscisa $0$.\\
\begin{indicacion}
\begin{enumerate}
\item En primer lugar se calcularía la derivada de $y$ con respecto a $x$, lo que ya se ha hecho en el ejercicio $2$.

\item A continuación se calcula el valor de $y$ correspondiente a $x=0$. Para ello se introduce $x:=0$ en la línea de edición, se selecciona
$x^{3}-3xy^{2}+y^{3}-1$ y se utiliza el menú \menu{Resolver > Expresión}, apareciendo un cuadro. En el campo \opcion{Variables} se deja $y$,
en el \opcion{Método} se selecciona \opcion{Algebraico}, en el \opcion{Dominio} se elige \opcion{Real} y se pincha en el botón
\boton{Resolver}.
\item A continuación se introduce $y:=$ al valor de $y$ obtenido en el punto anterior; y, una vez hecho esto, se selecciona la expresión
obtenida de la derivada y se utiliza el botón \boton{$\approx$} para obtener el valor de la derivada en el punto deseado. Con esto se
dispone de las coordenadas del punto y del valor de la derivada en dicho punto, y se puede escribir la ecuación de la tangente. Para anular
la asignación de valores a $x$ e $y$ se introduce en la línea de edición sucesivamente $x:= $ e $y= $.
\end{enumerate}
\end{indicacion}

\item Un punto se mueve en el plano siguiendo una trayectoria
\[\left\{
\begin{array}{l}
x=\sen t\\
y=t^{2}-1
\end{array}
\right.\]
Se pide:
\begin{enumerate}
\item Hallar la derivada de la función $y(x)$, es decir la derivada de $y$ con respecto a $x$, para $t=0$ y $t=2$.
\begin{indicacion}
\begin{enumerate}
\item Introducir sucesivamente en la línea de edición $x:=\sin{t}$, $y:=t^{2}-1$, $v:=[x,y]$ y la función \comando{PARA\_DIF(v,t,1)} para
calcular la derivada de $y$ con respecto a $x$ y pinchar en el símbolo \boton{=} para obtener el resultado.
\item A continuación se introduce $t:=0$ en la línea de edición, se selecciona la expresión de la derivada obtenida y se aplica el símbolo
\boton{$\approx$} para obtener el valor de la derivada en $t=0$, repitiéndose esto mismo para $t=2$. Una vez hecho esto escribir $t:= $ en
la línea de edición para eliminar las asignaciones de valores a $t$.
\end{enumerate}
\end{indicacion}

\item Hallar la ecuación de la tangente a la trayectoria en el punto $(0,-1)$
\begin{indicacion}
\begin{enumerate}
\item Para calcular el valor de $t$ que corresponde al citado punto, se utiliza el menú \menu{Resolver > Sistema}. En el primer cuadro que
aparece se introduce un $2$ en el campo \opcion{Número de Ecuaciones} y se pincha el botón \boton{Sí}, apareciendo otro cuadro en el que se
escribe $x=0$ e $y=-1$ en las líneas $1$ y $2$ respectivamente y se pincha en el campo \opcion{Variable} apareciendo en el mismo la variable
$t$. A continuación se pincha en \boton{Resolver} y se obtiene el valor de $t$ correspondiente al punto $(0,-1)$.
\item Conocido el valor de $t$ en el punto, podemos calcular fácilmente el valor de la derivada de $y$ con respecto a $x$ aprovechando la
derivada paramétrica obtenida en el primer apartado del problema, y con las coordenadas del punto y el valor de la derivada se puede
escribir la ecuación de la tangente. Una vez hecho esto, se anulan las asignaciones de valores realizadas de la misma forma que se hizo en
los ejercicios anteriores.
\end{enumerate}
\end{indicacion}
\end{enumerate}
\end{enumerate}


\section{Ejercicios propuestos}
\begin{enumerate}[leftmargin=*]
\item Derivar implícitamente la siguiente expresión tomando $y$ como función de $x$
\[
\ y=\frac{\sen(x+y)}{x^{2}+y^{2}}.
\]
\item Dada la función $xy+e^{x}-\log{y}=0$, calcular las rectas tangente y normal a ella en el punto $x=0$.

\item Una partícula se mueve a lo largo de una curva $y=\cos{2x+1}$ siendo $x=t^2+1$ y $t$ el tiempo. Calcular las componentes horizontal y vertical del vector velocidad cuando $t=2$.

\item  Una partícula se mueve a lo largo de la curva
\[\left\{
\begin{array}{l}
x=2\sen t\\
y=\sqrt{3}\cos{t}
\end{array}
\right.
\]
donde $x$ e $y$ están medidos en metros y el tiempo $t$ en segundos.
Se pide:
\begin{enumerate}
\item Hallar la ecuación de la recta tangente a la trayectoria en el punto (1,3/2).
\item ¿Cuáles son las componentes horizontal y vertical del vector velocidad en dicho punto?
\end{enumerate}
\end{enumerate}












