\documentclass[a4paper]{article}
\usepackage{svn-multi}
% Version control information:
\svnidlong
{$HeadURL: https://practicas-derive.googlecode.com/svn/trunk/discusion_sistemas_ecuaciones.tex $}
{$LastChangedDate: 2008-11-13 12:37:02 +0100 (jue, 13 nov 2008) $}
{$LastChangedRevision: 3 $}
{$LastChangedBy: asalber $}
\svnid{$Id: discusion_sistemas_ecuaciones.tex 3 2008-11-13 11:37:02Z asalber $}
\pdfinfo{/CreationDate (D:\svnpdfdate)}
\svnRegisterAuthor{alf}{Alfredo Sánchez Alberca}

\usepackage[spanish]{babel}
\usepackage[utf8x]{inputenc}
\usepackage{amsmath}
\usepackage{macros}
\usepackage[dvips]{graphicx}
\usepackage{enumitem}
\usepackage{subfigure}
\usepackage[small,bf]{caption2}
\usepackage[top=3cm, bottom=3cm, left=2.54cm, right=2.54cm]{geometry}
\usepackage{fancyhdr}
\pagestyle{fancy}

\lhead{\textsc{Universidad San Pablo CEU}} \rhead{\textsl{\textsf{Departamento de Métodos Cuantitativos}}}
\renewcommand{\headrulewidth}{0pt}
\renewcommand{\floatpagefraction}{.8}
\renewcommand{\textfraction}{.1}
\setcaptionwidth{\textwidth} \addtolength{\captionwidth}{-40pt}
\captionstyle{indent} \setlength\captionindent{\parindent}

\makeatletter
\let\savees@listquot\es@listquot
\def\es@listquot{\protect\savees@listquot}
\makeatletter

\begin{document}
\practica{Práctica de Álgebra con DERIVE}{Discusión de Sistemas de Ecuaciones Lineales}

\bigskip
\section*{Fundamentos Teóricos}

Los sistemas de ecuaciones lineales forman parte del corazón del álgebra y aparecen continuamente en multiples problemas de distintas disciplinas como la física, la química o la economía. En esta práctica se introduce el concepto de sistema de ecuaciones lineales, y se muestra cómo discutirlo atendiendo al número de soluciones que tenga.

\subsection*{Ecuaciones lineales y sistemas de ecuaciones}
Una \emph{ecuación lineal} con las variables $x_1,\ldots,x_n$ es una ecuación que puede escribirse de la forma
\[
a_1x_1+a_2x_2+\cdots+a_nx_n=b,
\]
donde $a_1,\ldots,a_n$ y $b$ son números reales o complejos conocidos como \emph{coeficientes} y \emph{término independiente} respectivamente. A las variables que intervienen en la ecuación se les llama \emph{incógnitas} de la ecuación.

Un sistema de ecuaciones lineales es una colección de ecuaciones lineales en las que intervienen las mismas variables $x_1,\ldots,x_n$, y que suele escribirse de la forma
\[\left\{
\begin{array}{c}
 a_{11}x_1+a_{12}x_2+\cdots+a_{1n}x_n=b_1 \\
 a_{21}x_1+a_{22}x_2+\cdots+a_{2n}x_n=b_2 \\
                  \vdots                  \\
 a_{m1}x_1+a_{m2}x_2+\cdots+a_{mn}x_n=b_m \\
\end{array}
\right.
\]

\subsubsection*{Notación Matricial}
Un sistema de ecuaciones lineales como el anterior puede representarse utilizando matrices mediante la ecuación matricial
\[\begin{array}{c}
\underbrace{\left(
\begin{array}{cccc}
 a_{11} & a_{12} & \cdots & a_{1n} \\
 a_{21} & a_{22} & \cdots & a_{2n} \\
 \vdots & \vdots & \ddots & \vdots \\
 a_{mn} & a_{m2} & \cdots & a_{mn} \\
\end{array}
\right)}\\
A
\end{array}
\begin{array}{c}
\underbrace{\left(
\begin{array}{c}
  x_1   \\
  x_2   \\
 \vdots \\
  x_ n  \\
\end{array}
\right)}\\
X
\end{array}
=
\begin{array}{c}
\underbrace{\left(
\begin{array}{c}
  b_1   \\
  b_2   \\
 \vdots \\
  b_m   \\
\end{array}
\right)}\\
B
\end{array}
\]

donde $A$ es la matriz de coeficientes, $X$ la matriz columna formada por las incógnitas y $B$ la matriz columna formada por los términos independientes.

Se llama \emph{solución del sistema} a un vector columna que colocado en el lugar de $X$ cumple la ecuación matricial $AX=B$.



\subsection*{Discusión de sistemas}
Un sistema de ecuaciones lineales puede no tener solución, o bien tener exactamente una solución o bien tener un número infinito de soluciones. Si un sistema tiene solución se denomina \emph{compatible} y si carece de solución \emph{incompatible}. Cuando la solución es única se llama \emph{compatible determinado} y si tiene varias soluciones \emph{compatible indeterminado}.

Discutir un sistema consiste en ver si tiene soluciones y cuántas, y suele hacerse de acuerdo al siguiente teorema.

\begin{teorema}[Rouché-Fröbenius]
Un sistema es compatible si y sólo si el rango de la matriz de coeficientes es igual al rango de dicha matriz ampliada con la columna de los términos independientes. 

En los sistemas compatibles, si el número de incógnitas es igual al rango de la matriz de coeficientes, el sistema se llama \emph{compatible determinado} y si es mayor que el rango \emph{compatible indeterminado}.
\end{teorema}


\section*{Ejercicios Prácticos}

Antes de hacer ejercicios relacionados con sistemas de ecuaciones es conveniente cargar el fichero \texttt{vector.mth} ya que se utilizarán funciones definidas en este fichero. 

\begin{enumerate}[leftmargin=*]
\item Dados los sistemas:

\[\left\{
\begin{array}{l}
x+2y=1 \\
2x-y=-2
\end{array}\right.
\qquad
\left\{
\begin{array}{l}
x-2y=-1 \\
-2x+4y=2
\end{array}\right.
\]

Se pide:
\begin{enumerate}
\item Introducir los sistemas en forma matricial.
\item Discutir los sistemas (utilizar la función \texttt{Rank} para calcular los rangos de la matriz de coeficiente y de la matriz ampliada con los términos independientes).
\item Para cada sistema, representar gráficamente las rectas que determinan cada una de las ecuaciones del sistema, y ver el número de soluciones del sistema mediante los puntos de corte de dichas rectas.
\end{enumerate}

\item Discutir los siguientes sistemas:
\[
\left\{
\begin{array}{l}
x+3y-2z=4 \\
2x+2y+z=3 \\
3x+2y+z=5 
\end{array}\right.
\qquad
\left\{
\begin{array}{l}
x-2y+z=0 \\
x+y-2z=2 \\
-2x+y+z=-2 
\end{array}\right.
\qquad
\left\{
\begin{array}{l}
x+y+2z=0 \\
-2x+y-z=-4 \\
3x-2y+z=-4 
\end{array}\right.
\]

\item Dado el sistema:

\[
\left\{
\begin{array}{l}
mx+y-z=0 \\
x+3y+z=0 \\
3x+10y+4z=0  
\end{array}\right.
\]
Se pide:
\begin{enumerate}
\item Introducir el sistema en forma matricial.
\item Discutir el sistema en función de los valores del parámetro $m$ (calcular el determinante de la matriz de coeficientes, y para que valores de $m$ que lo anulan, sustituir el parámetro por dichos valores y calcular los rangos de la matriz de coeficientes y la matriz ampliada).
\end{enumerate}

\item Discutir los siguientes sistemas en función de los valores de sus parámetros:

\[
\left\{
\begin{array}{l}
2x+my+4z=0 \\
x+y+7z=0 \\
mx-y+13z=0   
\end{array}\right.
\qquad
\left\{
\begin{array}{l}
2x+y  =1 \\
x+y-2z=1 \\
3x+y+mz=p   
\end{array}\right.
\]
\end{enumerate}




\section*{Problemas}
\begin{enumerate}[leftmargin=*]

\item Discutir los siguientes sistemas en función de los valores de sus parámetros:
\[
\left\{
\begin{array}{l}
x+y+2z=2 \\
2x-y+3z=2 \\
5x-y+mz=6  
\end{array}\right.
\qquad
\left\{
\begin{array}{l}
(m+1)x+y+z=3 \\
x+2y+mz=4 \\
x+my+2z=2  
\end{array}\right.
\qquad
\left\{
\begin{array}{l}
3x-y+2z=1 \\
x+4y+z=p \\
2x-5y+mz=-2  
\end{array}\right.
\] 
\end{enumerate}
\end{document}












