% Author: Alfredo Sánchez Alberca (asalber@ceu.es)
\chapter{Límites y Continuidad}

\section{Fundamentos teóricos}
En esta práctica se introducen los conceptos de límite y continuidad de una función real, ambos muy relacionados.

\subsection{Límite de una función en un punto}
El concepto de límite está muy relacionado con el de proximidad y tendencia de una serie de valores. De manera informal, diremos que $l\in \mathbb{R}$ es el \emph{límite} de una función $f(x)$ en un punto $a\in \mathbb{R}$, si $f(x)$ tiende o se aproxima cada vez más a $l$, a medida que $x$ se aproxima a $a$, y se escribe
\[ \lim_{x\rightarrow a} f(x)=l.\]

Si lo que nos interesa es la tendencia de $f(x)$ cuando nos aproximamos al punto $a$ sólo por un lado, hablamos de \emph{límites laterales}. Diremos que $l$ es el \emph{límite por la izquierda} de una función $f(x)$ en un punto $a$, si $f(x)$ tiende o se aproxima cada vez más a $l$, a medida que $x$ se aproxima a $a$ por la izquierda, es decir con valores $x<a$, y se denota por
\[ \lim_{x\rightarrow a^-} f(x)=l.\]
Del mismo modo, diremos que $l$ es el \emph{límite por la derecha} de una función $f(x)$ en un punto $a$, si $f(x)$ tiende o se aproxima cada vez más a $l$, a medida que $x$ se aproxima a $a$ por la derecha, es decir con valores $x>a$, y se denota por
\[ \lim_{x\rightarrow a^+} f(x)=l.\]

Por supuesto, para que exista el límite global de la función $f(x)$ en el punto $a$, debe existir tanto el límite por la izquierda, como el límite por la derecha, y ser iguales, es decir
\[
\left.
\begin{array}{l}
\displaystyle \lim_{x\rightarrow a^-} f(x)=l\\
\displaystyle \lim_{x\rightarrow a^+} f(x)=l
\end{array}
\right\}
\Longrightarrow
\lim_{x\rightarrow a} f(x)=l.
\]

\subsection{Álgebra de límites}
Para el cálculo práctico de límites, se utiliza el siguiente
teorema, conocido como Teorema de \emph{Álgebra de Límites}.

Dadas dos funciones $f(x)$ y $g(x)$, tales que $\lim_{x\rightarrow
a}f(x)=l_1$ y $\lim_{x\rightarrow a}g(x)=l_2$, entonces se cumple
que:
\begin{enumerate}
\item $\displaystyle \lim_{x\rightarrow a}(f(x)\pm g(x))=l_1\pm l_2$.
\item $\displaystyle \lim_{x\rightarrow a}(f(x)\cdot g(x))=l_1\cdot l_2$.
\item $\displaystyle \lim_{x\rightarrow a}\dfrac{f(x)}{g(x)}=\dfrac{l_1}{l_2}$ si $l_2\neq 0$.
\end{enumerate}

\subsection{Asíntotas}
Como interpretación geométrica de los límites, definiremos rectas
particulares a las que tiende (se ``pega") la gráfica de una función
cuando la variable tiende a un cierto valor, finito o infinito.
\subsubsection*{Asíntotas verticales}
La recta $x=a$ es una \emph{Asíntota Vertical} de la función $f(x)$
si al menos uno de los límites laterales de $f$ en $a$ es $+\infty$
ó $+\infty$. Es decir:

\[
\mathop {\lim }\limits_{x \to a} f(x) =  \pm \infty
\]

\subsubsection*{Asíntotas Horizontales}
La recta $y=b$ es una \emph{Asíntota Horizontal} de la función
$f(x)$ si se cumple:
\[
\mathop {\lim }\limits_{x \to  + \infty } f(x) = b\quad
\text{ó}\quad\mathop {\lim }\limits_{x \to  - \infty } f(x) = b
\]


\subsubsection*{Asíntotas Oblicuas}

La recta $y=mx+n$, donde $m\neq0$, es \emph{Asíntota Oblicua} de la
función $f(x)$ si:


\[
\mathop {\lim }\limits_{x \to  + \infty } \left[ {f(x) - \left( {mx
+ n} \right)} \right] = 0\quad\text{ó}\quad\mathop {\lim }\limits_{x
\to - \infty } \left[ {f(x) - \left( {mx + n} \right)} \right] = 0
\]


La determinación práctica de $m$ y $n$ se realiza del siguiente
modo:

\[
m = \mathop {\lim }\limits_{x \to  + \infty } \frac{{f(x)}} {x}
\]

\[
n = \mathop {\lim }\limits_{x \to  + \infty } \left[ {f(x) - mx}
\right]
\]
o bien lo mismo con los límites en $-\infty$:
\[
m = \mathop {\lim }\limits_{x \to  - \infty } \frac{{f(x)}} {x}
\]

\[
n = \mathop {\lim }\limits_{x \to  - \infty } \left[ {f(x) - mx}
\right]
\]

En cualquiera de los casos, si obtenemos un valor real para $m$ (no
puede ser ni $+\infty$ ni $-\infty$) distinto de $0$, procedemos
después a calcular $n$, que también debe ser real (sí que puede ser
$0$).

Si $m=\pm\infty$ entonces la función crece (decrece) más deprisa que
cualquier recta, y si $m=0$ la función crece (decrece) más despacio
que cualquier recta, y en cualquiera de los dos casos decimos que la
función tiene una \emph{Rama Parabólica}.

\subsection{Continuidad de una función en un punto}
Diremos que una función $f(x)$ es continua en un punto $a\in
\mathbb{R}$, si se cumple
\[ \lim_{x\rightarrow a}f(x)=f(a),\]
donde $f(a)\in \mathbb{R}$.

La definición anterior implica a su vez que se cumplan estas tres
condiciones:

\begin{itemize}

\item Existe el límite de $f$ en $x=a$.

\item La función está definida en $x=a$; es decir, existe $f(a)$.

\item Los dos valores anteriores coinciden.

\end{itemize}

Si la función $f$ no es continua en $x=a$, diremos que es
\emph{discontinua} en el punto $a$, o bien que $f$ tiene una
\emph{discontinuidad} en $a$.

Intuitivamente, una función es continua cuando puede dibujarse su
gráfica sin levantar el lápiz.

\subsubsection*{Continuidad lateral en un punto}

Si nos restringimos a los valores que toma una función a la derecha
de un punto $x=a$, o a la izquierda, se habla de continuidad por la
derecha o por la izquierda según la siguiente definición.

Una función es \emph{continua por la derecha} en un punto $x=a$, y
lo notaremos como $f$ continua en $a^+$, si existe el límite por la
derecha en dicho punto y coincide con el valor de la función en el
mismo:
\[
\mathop {\lim }\limits_{x \to a^ +  } f\left( x \right) = f\left( a
\right)
\]

De igual manera, la función es \emph{continua por la izquierda} en
un punto $x=a$, y lo notaremos como $f$ continua en $a^-$, si existe
el límite por la izquierda en dicho punto y coincide con el valor de
la función en el mismo:

\[
\mathop {\lim }\limits_{x \to a^ -  } f\left( x \right) = f\left( a
\right)
\]


\subsubsection*{Propiedades de la continuidad en un punto}

Como consecuencia de la definición de continuidad en un punto,
podrían demostrarse toda una serie de teoremas, algunos de ellos
especialmente importantes.

\begin{itemize}

\item \textbf{Álgebra de funciones continuas}.
Si $f$ y $g$ son funciones continuas en $x=a$, entonces $f\pm g$ y
$f\cdot g$ son también continuas en $x=a$. Si además $g(a)\neq 0$,
entonces $f/g$ también es continua en $x=a$.

\item \textbf{Continuidad de funciones compuestas}. Si $f$ es continua en
$x=a$ y $g$ es continua en $b=f(a)$, entonces la función compuesta
$g\circ f$ es continua en $x=a$.

\item \textbf{Continuidad y cálculo de límites}. Sean $f$ y $g$ dos
funciones tales que existe $\mathop {\lim }\limits_{x \to a} f(x) =
l$ $\in \mathbb{R}$ y $g$ es una función continua en $l$. Entonces:

\[
\mathop {\lim }\limits_{x \to a} g\left( {f\left( x \right)} \right)
= g\left( l \right)
\]

\end{itemize}

\subsubsection*{Tipos de discontinuidades}
Puesto que la condición de continuidad puede no satisfacerse por
distintos motivos, existen distintos tipos de discontinuidades:


\begin{itemize}
\item \textbf{Discontinuidad evitable}. Se dice que $f(x)$ tiene una \emph{discontinuidad evitable} en el punto $a$, si existe el límite de la función  pero no coincide con el valor de la función en el punto (bien porque sea diferente, bien por que la función no esté definida en dicho punto), es decir
\[\lim_{x\rightarrow a}f(x)=l\neq f(a).\]

\item \textbf{Discontinuidad de salto}. Se dice que $f(x)$ tiene una \emph{discontinuidad de salto} en el punto $a$, si existe el límite de la función por la izquierda  y por la derecha pero son diferentes, es decir,
\[
\lim_{x\rightarrow a^-}f(x)=l_1\neq l_2=\lim_{x\rightarrow a^+}f(x).
\]
A la diferencia entre ambos límites $l_1-l_2$, se le llama
\emph{amplitud del salto}.

\item \textbf{Discontinuidad esencial}. Se dice que $f(x)$ tiene una \emph{discontinuidad esencial} en el punto $a$, si no existe alguno de los límites laterales de la función.
\end{itemize}

\newpage

\section{Ejercicios resueltos}
\begin{enumerate}[leftmargin=*]
\item  Dada la función
\[
f(x)=\left( 1+\frac 2x\right) ^{x/2},
\]
se pide:

\begin{enumerate}
\item  Dibujar su gráfica, y a la vista de misma conjeturar el resultado de los siguientes límites:
\begin{multicols}{2}
\begin{enumerate}
\item  $\lim\limits_{x\rightarrow -\,\infty }\ f(x)$
\item  $\lim\limits_{x\rightarrow +\,\infty }\ f(x)$
\item  $\lim\limits_{x\rightarrow -\,2^{-}}\ f(x)$
\item  $\lim\limits_{x\rightarrow -\,2^{+}}\ f(x)$
\item  $\lim\limits_{x\rightarrow 2}\ f(x)$
\item  $\lim\limits_{x\rightarrow 0}\ f(x)$
\end{enumerate}
\end{multicols}

\begin{indicacion}
\begin{enumerate}
\item Para representar la gráfica de la función, introducir su expresión, acceder a la ventana 2D, y pinchar en el botón \boton{Representar
Expresión}. Probablemente haya que cambiar la escala de la representación original pinchando en el botón de \boton{Zoom hacia fuera en ambos
ejes}, para tener una perspectiva más amplia de la forma de la función.
\item Para predecir cuáles pueden ser los valores de los límites pedidos, observar hacia qué valores tiende la función cuando la variable
$x$ se acerca al valor que aparece en cada límite. Para ello, puede resultar conveniente pinchar en el botón \boton{Trazar gráficas}, para
que el cursor sólo pueda desplazarse a lo largo de la misma.
\end{enumerate}
\end{indicacion}

\item  Calcular los límites anteriores. ¿Coinciden los resultados con los conjeturados?.
\begin{indicacion}
\begin{enumerate}
\item Utilizar el menú \menu{Cálculo > Límites}, o su correspondiente botón de la barra de botones.

\item En el cuadro de diálogo que aparece, seleccionar tanto la variable del límite como el punto en el que queremos calcularlo, y la
tendencia (izquierda, derecha, o ambas).
\end{enumerate}
\end{indicacion}
\end{enumerate}


\item Para la función $\ln(x^2-1)$, se pide: 
\begin{enumerate}
\item Representar la gráfica. A la vista de la gráfica, ¿que tipo de asíntotas crees que tiene la función?
\begin{indicacion}
\begin{enumerate}
\item Para representar la gráfica, seguimos el proceso comentado en apartados anteriores (introducir la expresión, acceder a la ventana 2D,
y utilizar el botón \boton{Representar Expresiones}).
\item Para determinar, de forma gráfica, la presencia de asíntotas, hay que ver si la gráfica tiende a una forma rectilínea en $\pm \infty$,
con lo que tendríamos una asíntota horizontal (tendencia a una recta horizontal), u oblicua (si la tendencia es a una recta de pendiente no
nula); y también ver si la gráfica tiende a parecerse a una recta vertical en algún valor concreto de la variable, con lo cual tendríamos
una asíntota vertical en dicho punto.
\end{enumerate}
\end{indicacion}

\item Calcular los límites cuando $x$ tiende a $1$ por la derecha
y a $-1$ por la izquierda para demostrar la presencia de asíntotas
verticales.

\begin{indicacion}
Utilizar el botón \boton{Calcular un Límite}, y escoger la variable, el punto y la tendencia. Si alguno de los límites realizados en igual a
$+\infty$ ó $-\infty$, entonces sí que hay una asíntota vertical en los puntos considerados.
\end{indicacion}

\item Calcular los límites cuando $x$ tiende a $\pm\infty$ para analizar la presencia de asíntotas horizontales. ¿Qué se concluye a la vista
de los resultados obtenidos?

\begin{indicacion}
\begin{enumerate}
\item Para calcular los límites, proceder de la misma forma que en el punto anterior pero escogiendo el $\infty$ de la barra de operadores y
constantes. En cuanto a la tendencia, a pesar de que a $+\infty$ sólo se puede tender desde la izquierda, y a $-\infty$ desde la derecha, en
los límites en $\pm\infty$ da igual la forma de tendencia que escojamos.
\item En cuanto a la presencia o no de asíntotas horizontales, es necesario que los límites anteriores valgan 0.
\end{enumerate}
\end{indicacion}

\item Calcular los límites cuando $x$ tiende a $\pm\infty$ de la función dividida entre $x$ para analizar la presencia de asíntotas
oblicuas. ¿Qué se concluye a la vista de los resultados obtenidos?.
\begin{indicacion}
Recordar que, en caso de que la función tenga asíntotas oblicuas, la pendiente de la asíntota en $+\infty$ vendría dada por:
\[
m = \mathop {\lim }\limits_{x \to  + \infty } \frac{{f(x)}} {x}
\]
e igual en $-\infty$, cambiando el $+\infty$ por $-\infty$.

Por lo tanto, se divide la función entre $x$ y se procede a calcular los límites anteriores mediante el botón \boton{Calcular un límite}.
Si los límites obtenidos valen 0, quiere decir que no hay asíntotas oblicuas, que la tendencia creciente de la función es menos marcada que
la de cualquier recta con pendiente no nula.
\end{indicacion}
\end{enumerate}

\item  Clasificar las discontinuidades de las siguientes funciones en los puntos que se indica.
\begin{multicols}{2}
\begin{enumerate}
\item  $f(x)=\dfrac{\sen x}{x}$ en $x=0$.
\item $g(x)=\dfrac{1}{2^{1/x}}$ en $x=0$.
\item $h(x)=\dfrac{1}{1+e^{\frac{1}{1-x}}}$ en $x=1$.
\end{enumerate}
\end{multicols}

\begin{indicacion}
Para clasificar las discontinuidades en los puntos que se indican, además de definir cada una de las funciones, conviene representar su
gráfica, lo cual, aunque no sirve para demostrar la presencia de una discontinuidad, sí que nos puede dar una idea sobre las
discontinuidades presentes y su tipo (las discontinuidades evitables apenas aparecen visibles en la gráfica, aunque sí que Derive deja un
pequeño hueco en la misma):
\begin{enumerate}
\item Calcular el valor del límite en el punto. Para ello, se puede utilizar el botón \boton{Calcular un límite} de la barra de botones, y
activar la tendencia por \opcion{Ambas} en el cuadro de diálogo que aparece. Si dicho límite existe, entonces la discontinuidad es evitable.
\item Si el límite no existe, puede que sí que existan los laterales. Para calcularlos utilizar el botón \boton{Calcular un límite} y
activar la tendencia por la \opcion{Izquierda} y luego por la \opcion{Derecha}. Si ambos límites laterales existen pero no son iguales, la
discontinuidad será de salto.
\item Si alguno de los límites laterales no existe, entonces la discontinuidad es esencial.
\end{enumerate}
\end{indicacion}


\item  Hallar los puntos de discontinuidad y estudiar el carácter
de dichas discontinuidades en la función:
\[
f(x)=
\left\{
\begin{array}{ll}
\dfrac{x+1}{x^2-1}, & \hbox{si $x<0$;} \\
\dfrac{1}{e^{1/(x^2-1)}}, & \hbox{si $x\geq 0$.} \\
\end{array}
\right.
\]

\begin{indicacion}
\begin{enumerate}
\item Para delimitar los posibles puntos de discontinuidad, previamente definir la función teniendo en cuenta que se trata de una función
definida a trozos. Por lo tanto, habrá que multiplicar el primer tramo por \comando{CHI}$(\infty,x,0)$, y el segundo por
\comando{CHI}$(0,x,\infty)$.
\item Aunque no sirve para demostrar la presencia o no de una discontinuidad, si representamos la gráfica de la función podemos darnos una
idea sobre los puntos en los que aparecen las discontinuidades, teniendo muy presente que las discontinuidades evitables apenas resultan
visibles en la gráfica, aunque sí que Derive deja un pequeño hueco en la misma.
\item Una vez definida la función, hay que encontrar los puntos que quedan fuera del dominio de cada uno de los tramos. Para ello, hay que
analizar dónde se anulan los denominadores presentes en las definiciones de ambos tramos. Por ejemplo, si $x<0$, el denominador $x^2-1$ se
anula en $x=\pm1$; sin embargo tan sólo nos interesa $x=-1$ ya que la definición impone que $x<0$.
\item Cuando ya hemos descubierto cuáles son los puntos que están fuera del dominio, y por tanto son discontinuidades de la función, hay que
analizar cual es su tipo. Para ello, aplicamos el mismo proceso que en el ejercicio anterior (vemos si existe el límite, con lo cual sería
discontinuidad evitable, y si no existe analizamos los laterales para ver si es discontinuidad de salto; si no existe alguno de los
laterales es discontinuidad esencial).
\item Por último, también hay que analizar los puntos en los que hay un cambio de definición de la función. En nuestro caso, en $x=0$, y, de
nuevo, analizando el límite general y los límites laterales.
\end{enumerate}
\end{indicacion}

\end{enumerate}


\section{Ejercicios propuestos}
\begin{enumerate}[leftmargin=*]
\item  Calcular los siguientes límites si existen:
\begin{multicols}{2}
\begin{enumerate}
\item  $\displaystyle \lim_{x\rightarrow 1}\dfrac{x^3-3x+2}{x^4-4x+3}$.
\item  $\displaystyle \lim_{x\rightarrow a}\dfrac{\sen x-\sen a}{x-a}$.
\item $\displaystyle \lim_{x\rightarrow\infty}\dfrac{x^2-3x+2}{e^{2x}}$.
\item $\displaystyle \lim_{x\rightarrow\infty}\dfrac{\log(x^2-1)}{x+2}$.
\item $\displaystyle \lim_{x\rightarrow 1}\dfrac{\log(1/x)}{\tg(x+\dfrac{\pi}{2})}$.
\item $\displaystyle \lim_{x\rightarrow a}\dfrac{x^n-a^n}{x-a}\quad n\in \mathbb{N}$.
\item $ \displaystyle \lim_{x\rightarrow 1}\dfrac{\sqrt[n]{x}-1}{\sqrt[m]{x}-1}\quad n,m \in \mathbb{Z}$.
\item $\displaystyle \lim_{x\rightarrow 0}\dfrac{\tg x-\sen x}{x^3}$.
\item $\displaystyle \lim_{x\rightarrow \pi/4}\dfrac{\sen x-\cos x}{1-\tg x}$.
\item $\displaystyle \lim_{x\rightarrow 0}x^2e^{1/x^2}$.
\item $\displaystyle \lim_{x\rightarrow \infty}\left(1+\dfrac{a}{x}\right)^x$.
\item $\displaystyle \lim_{x\rightarrow \infty} \sqrt[x]{x^2}$.
\item $\displaystyle \lim_{x\rightarrow 0}\left(\dfrac{1}{x}\right)^{\tg x}$.
\item $\displaystyle \lim_{x\rightarrow 0}(\cos x)^{1/\mbox{\footnotesize sen}\, x}$.
\item $\displaystyle \lim_{x\rightarrow 0}\dfrac{6}{4+e^{-1/x}}$.
\item $\displaystyle \lim_{x\rightarrow \infty}\left(\sqrt{x^2+x+1}-\sqrt{x^2-2x-1}\right)$.
\item $\displaystyle \lim_{x\rightarrow \pi/2}\sec x-\tg x$.
\end{enumerate}
\end{multicols}

\item Dada la función:
\[
\renewcommand{\arraystretch}{2}
f(x) = \left\{
\begin{array}{*{20}c}{
\dfrac{{x^2  + 1}}{{x + 3}}} & {\text{si}\;x < 0}  \\
{\dfrac{1}{{e^{1/(x^2  - 1)} }}} & {\text{si}\;x \geqslant 0\;}  \\
\end{array} 
\right.
\]
Calcular todas sus asíntotas.

\item  Las siguientes funciones no están definidas en $x=0$.
Determinar, cuando sea posible, su valor en dicho punto de modo que sean continuas.
\begin{multicols}{2}
\begin{enumerate}
\item  $f(x)=\dfrac{(1+x)^n-1}{x}$.
\item  $h(x)=\dfrac{e^x-e^{-x}}{x}$.
\item  $j(x)=\dfrac{\log(1+x)-\log(1-x)}{x}$.
\item  $k(x)=x^2\sen\dfrac{1}{x}$.
\end{enumerate}
\end{multicols}

\end{enumerate}
