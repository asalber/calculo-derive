% Author: Alfredo Sánchez Alberca (asalber@ceu.es)
\chapter{Integrales}

\section{Fundamentos teóricos}
Junto al concepto de derivada, el de integral es otro de los más importantes
del cálculo matemático. Aunque dicho concepto surge en principio, como técnica
para el cálculo de áreas, el teorema fundamental del cálculo establece su
relación con la derivada, de manera que, en cierto sentido, la diferenciación
y la integración son operaciones inversas.

En esta práctica se introduce el concepto de integral como antiderivada, y
también el de integral de Riemann, que permite calcular áreas por debajo de
funciones acotadas en un intervalo.

\subsection{Primitivas e Integrales}
\subsubsection*{Función Primitiva}

Se dice que la función $F(X)$ es una \emph{función primitiva} de
$f(x)$ si se verifica que $F'(x)=f(x)$ $\forall x \in \dom f$.

Como dos funciones que difieran en una constante tienen la misma
derivada, si $F(x)$ es una función primitiva de $f(x)$ también lo será toda función de la forma $F(x)+k$ $\forall k \in \mathbb{R}$.\\


\subsubsection*{Función integral indefinida}

Se llama \emph{función integral indefinida} de la función $f(x)$ al
conjunto de todas sus funciones primitivas y se representa como:

\[
\ \int{f(x)}\,dx=F(x)+C
\]
siendo $F(x)$ una función primitiva de $f(x)$ y $C$ una constante arbitraria.\\


\subsubsection*{Linealidad de la integral}

Dadas dos funciones $f(x)$ y $g(x)$ que admiten primitiva, y una
constante $k \in \mathbb{R}$ se verifica que:

\[
\ \int{(f(x)+g(x))}\,dx=\int{f(x)}\,dx+\int{g(x)}\,dx
\]
y:
\[
\ \int{kf(x)}\,dx=k\int{f(x)}\,dx
\]\\


\subsection{Integral de Riemann}

Se llama \emph{partición} de un intervalo $[a,b]\subset\mathbb{R}$,
a una colección finita de puntos del intervalo,
$P=\{x_{0},x_{1},...,x_{n}\}$,  tales que
$x_{0}=a<x_{1}<...<x_{n}=b$, con lo que el intervalo $[a,b]$ queda
dividido en $n$ subintervalos $[x_{i},x_{i+1}]$, $i=0,...,n-1$.

Dada una función $f:[a,b]\rightarrow\mathbb{R}$ acotada y una
partición $P=\{x_{0},x_{1},...,x_{n}\}$ de $[a,b]$, se llama
\emph{suma inferior} de $f$ en relación a $P$, y se designa por
$L(P,f)$, a:

\[
\ L(P,f)=\sum_{i=1}^{n} m_{i}(x_{i}-x_{i-1})
\]
donde $  m_{i}=\inf\{f(x):x_{i-1}\leq x \leq x_{i}\}$.

Análogamente se llama \emph{suma superior} de $f$ en relación a $P$,
y se designa por $U(P,f)$, a:

\[
\ U(P,f)=\sum_{i=1}^{n} M_{i}(x_{i}-x_{i-1})
\]
donde $ M_{i}=\sup\{f(x):x_{i-1}\leq x \leq x_{i}\}$.

La \emph{suma inferior} y la \emph{suma superior} así definidas
representan las sumas de las áreas de los rectángulos que tienen por
bases los subintervalos de la partición, y por alturas los valores
mínimo y máximo respectivamente de la función $f$ en los
subintervalos considerados, tal y como se muestra en la
figura~\ref{g:sumassupinf}. Así, los valores de $L(P,f)$ y $U(P,f)$
serán siempre menores y mayores respectivamente, que el área
encerrada por la función $f$ y el eje de abscisas en el intervalo
$[a,b]$.

\begin{figure}[htbp]
\centering \subfigure[Suma inferior $L(P,f)$.]{
\label{g:sumainferior}
\scalebox{1}{\psset{unit=0.6,ticksize=-3pt 0,plotpoints=200}
\begin{pspicture}(-0.1,-1)(10,6)
\psaxes[arrows=<->,ticks=none,labels=none]{<->}(0,0)(-1,-1)(10,6)
\psStep[algebraic,linecolor=gray,StepType=infimun,fillstyle=solid,fillcolor=coral](1,9){8}{sqrt(x)*sin(x)+3}
\psplot[algebraic,linecolor=blue]{0}{10}{sqrt(x)*sin(x)+3}
\psxTick[ticksize=-4pt 0](1){a}
\psxTick[ticksize=-4pt 0](2){x_1}
\psxTick[ticksize=-4pt 0](8){x_{n-1}}
\psxTick[ticksize=-4pt 0](9){b}
\rput[t](5,-0.3){$\cdots$}
\rput[r](10,5){$f(x)$}
\end{pspicture}}}\qquad\qquad
\subfigure[Suma superior$U(P,f)$.]{
\label{g:sumasuperior}
\scalebox{1}{\psset{unit=0.6,ticksize=-3pt 0,plotpoints=200}
\begin{pspicture}(-0.1,-1)(10,6)
\psaxes[arrows=<->,ticks=none,labels=none]{<->}(0,0)(-1,-1)(10,6)
\psStep[algebraic,linecolor=gray,StepType=supremun,fillstyle=solid,fillcolor=blueceu!50](1,9){8}{sqrt(x)*sin(x)+3}
\psplot[algebraic,linecolor=blue]{0}{10}{sqrt(x)*sin(x)+3}
\psxTick[ticksize=-4pt 0](1){a}
\psxTick[ticksize=-4pt 0](2){x_1}
\psxTick[ticksize=-4pt 0](8){x_{n-1}}
\psxTick[ticksize=-4pt 0](9){b}
\rput[t](5,-0.3){$\cdots$}
\rput[r](10,5){$f(x)$}
\end{pspicture}}}
\caption{Áreas medidas por las sumas superior e inferior
correspondientes a una partición.} \label{g:sumassupinf}
\end{figure}

Una función $f:[a,b]\rightarrow\mathbb{R}$ acotada es
\emph{integrable} en el intervalo $[a,b]$ si se verifica que:

\[
\ \sup\{L(P,f): P \textrm{ partición de } [a,b]\}=\inf\{U(P,f): P
\textrm{ partición de }[a,b]\}
\]
y ese número se designa por $\int_{a}^{b}f(x)\,dx$ o simplemente por
$\int_{a}^{b}f$.


\subsubsection*{Propiedades de la Integral}

\begin{enumerate}

\item \textbf{Linealidad}

Dadas dos funciones $f$ y $g$ integrables en $[a,b]$ y $k \in
\mathbb{R}$ se verifica que:

\[
\
\int_{a}^{b}(f(x)+g(x))\,dx=\int_{a}^{b}f(x)\,dx+\int_{a}^{b}g(x)\,dx
\]
y
\[
\ \int_{a}^{b}{kf(x)}\,dx=k\int_{a}^{b}{f(x)}\,dx
\]

\item \textbf{Monotonía}

Dadas dos funciones $f$ y $g$ integrables en $[a,b]$ y tales que
$f(x)\leq g(x)$ $\forall x \in [a,b]$ se verifica que:


\[
\ \int_{a}^{b}{f(x)\,dx} \leq \int_{a}^{b}{g(x)\,dx}
\]

\item \textbf{Acotación}

Si $f$ es una función integrable en el intervalo $[a,b]$, existen
$m,M\in\mathbb{R}$ tales que:

\[
\ m(b-a)\leq\int_{a}^{b}{f(x)\,dx} \leq \ M(b-a)
\]

\item \textbf{Aditividad}

Si $f$ es una función acotada en $[a,b]$ y $c\in(a,b)$, entonces $f$
es integrable en $[a,b]$ si y sólo si lo es en $[a,c]$ y en $[c,b]$,
verificándose además:

\[
\ \int_{a}^{b}{f(x)\,dx} =
\int_{a}^{c}{f(x)\,dx}+\int_{c}^{b}{f(x)\,dx}
\]\\

\end{enumerate}

\subsubsection*{Teorema Fundamental del Cálculo}

Sea $f : [a,b]\rightarrow\mathbb{R}$ continua y sea $G$ una función
continua en $[a,b]$. Entonces $G$ es derivable en $(a,b)$ y
$G'(x)=f(x)$ para todo $x\in(a,b)$ si y sólo si:

\[
\ G(x)-G(a) = \int_{a}^{x}f(t)\,dt
\]

\subsubsection*{Regla de Barrow}

Si $f$ es una función continua en $[a,b]$ y $G$ es continua en
$[a,b]$, derivable en $(a,b)$ y tal que $G'(x)=f(x)$ para todo
$x\in(a,b)$ entonces:

\[
\  \int_{a}^{b}{f} = G(b)-G(a)
\]


De aquí se deduce que:

\[
\  \int_{a}^{b}{f} = -\int_{b}^{a}{f}
\]


\subsection{Integrales impropias}

La integral $ \int_{a}^{b}{f(x)\,dx}$ se llama \emph{impropia} si el
intervalo $(a,b)$ no está acotado o si la función $f(x)$ no está
acotada en el intervalo $(a,b)$.

Si el intervalo $(a,b)$ no está acotado, se denomina integral
impropia de primera especie mientras que si la función no está
acotada en el intervalo se denomina integral impropia de segunda
especie.

\subsection{Cálculo de áreas}
Una de las principales aplicaciones de la integral es el cálculo de
áreas.

\subsubsection*{Área de una región plana encerrada por dos curvas}

Si $f$ y $g$ son dos funciones integrables en el intervalo $[a,b]$ y
se verifica que $g(x)\leq f(x)$ $\forall x\in[a,b]$, entonces el
área de la región plana limitada por las curvas $y=f(x)$, $y=g(x)$,
y las rectas $x=a$ y $x=b$ viene dada por:

\[
\ A = \int_{a}^{b}{(f(x)- g(x))\,dx}
\]\\

\noindent \textbf{Observaciones}

\begin{enumerate}

\item El intervalo $(a,b)$ puede ser infinito y la definición sería análoga, pero en ese caso es preciso que la integral impropia sea convergente.

\item Si $f(x)\geq0$ y $g(x)=0$ al calcular la integral entre $a$ y $b$ se obtiene el área encerrada por la función $f(x)$ y el eje de abscisas entre las rectas verticales $x=a$ y $x=b$ (figura~\ref{g:integral_definida}).

\begin{figure}[h!]
\begin{center}
\scalebox{1}{\psset{xunit=1.5, ticksize=-3pt 0}
\begin{pspicture}(-0.5,-0.5)(4.5,4.5)
\psaxes[arrows=<->,ticks=none,labels=none](0,0)(-0.5,-0.5)(4.5,4.5)
\pscustom[linecolor=blue]{%
\psplot{1}{3}{x 3 exp x 2 exp 6 mul sub 11 x mul add 3 sub}
\gsave
\psline(3,0)(1,0)
\fill[fillstyle=solid,fillcolor=blueceu!50]
\grestore
\psplot{3}{3.2}{x 3 exp x 2 exp 6 mul sub 11 x mul add 3 sub}}
\psplot[linecolor=blue]{0.6}{1}{x 3 exp x 2 exp 6 mul sub 11 x mul add 3 sub}
\psline[linecolor=gray](1,0)(1,3)
\psline[linecolor=gray](3,0)(3,3)
\psxTick(1){a}
\psxTick(3){b}
\rput[c](2,1.5){$\displaystyle \int_a^b f(x)\,dx$}
\rput[l](3.5,3.5){$f(x)$}
\end{pspicture}}
\caption{Cálculo de área encerrada por la función $f(x)$ y el eje de
abscisas entre las rectas verticales $x=a$ y $x=b$  mediante la
integral definida.} \label{g:integral_definida}
\end{center}
\end{figure}

\item Si $f(x)\geq 0$ $\forall x\in[a,c]$ y $f(x)\leq 0$ $\forall x\in[c,b]$, el área de la región plana encerrada por $f$, las rectas verticales $x=a$ y $x=b$ y el eje de abscisas se calcula
mediante:
\[
\ A= \int_{a}^{c}{f(x)\,dx} - \int_{c}^{b}{f(x)\,dx}.
\]

\item Si las curvas $y=f(x)$ e $y=g(x)$ se cortan en los puntos de abscisas $a$ y $b$, no cortándose en ningún otro punto cuya abscisa esté comprendida entre $a$ y $b$, el área encerrada por dichas curvas entre esos puntos de corte puede calcularse
mediante:
\[
\ A= \int_{a}^{b}{|f(x)-g(x)|dx}
\]
\end{enumerate}


\subsection{Cálculo de Volúmenes}

\subsubsection*{Volumen de un sólido}
Si se considera un cuerpo que al ser cortado por un plano
perpendicular al eje $OX$ da lugar, en cada punto de abscisa $x$, a
una sección de área $A(x)$, el volumen de dicho cuerpo comprendido
entre los planos perpendiculares al eje $OX$ en los puntos de
abscisas $a$ y $b$ es:

\[
\ V = \int_{a}^{b}{A(x)\,dx}
\]

\subsubsection*{Volumen de un cuerpo de revolución}
Si se hace girar la curva $y=f(x)$ alrededor del eje $OX$ se genera
un sólido de revolución cuyas secciones perpendiculares al eje $OX$
tienen áreas $A(x)=\pi(f(x))^{2}$, y cuyo volumen comprendido entre
las abscisas $a$ y $b$ será:

\[
\ V = \int_{a}^{b}{\pi(f(x))^{2}\,dx}=
\pi\int_{a}^{b}{(f(x))^{2}\,dx}
\]


En general, el volumen del cuerpo de revolución engendrado al girar
alrededor del eje $OX$ la región plana limitada por las curvas
$y=f(x)$, $y=g(x)$ y las rectas verticales $x=a$ y $x=b$ es:

\[
\ V = \int_{a}^{b}{\pi|(f(x))^{2}-(g(x))^{2}|\,dx}
\]

De manera análoga se calcula el volumen del cuerpo de revolución
engendrado por la rotación de una curva $x=f(y)$ alrededor del eje
$OY$, entre los planos $y=a$ e $y=b$, mediante:

\[
\ V = \int_{a}^{b}{\pi(f(y))^{2}dy} = \pi \int_{a}^{b}{(f(y))^{2}dy}
\]


\section{Ejercicios resueltos}

\begin{enumerate}[leftmargin=*]
\item Calcular las siguientes integrales:
\begin{enumerate}
\item $ \int{x^{2} \log{x}\,dx}$
\begin{indicacion}
Introducir \verb|x\^2log(x)| en la línea de edición. Utilizar el menú \menu{Cálculo > Integrales}, eligiendo \opcion{Integral Indefinida},
\opcion{Constante} $c$ y pinchar en el botón \boton{Simplificar}.
\end{indicacion}

\item $ \dint{\dfrac{5x^{2}+4x+1}{x^{5}-2x^{4}+2x^{3}-2x^{2}+x}\,dx}$
\begin{indicacion}
Introducir la expresión \verb|(5x^2+4x+1)/(x^5-2x^4+2x^3-2x^2+x)| en la línea de edición y proceder de la forma indicada en el apartado
anterior.
\end{indicacion}

\item $ \dint{\dfrac{6x+5}{(x^{2}+x+1)^{2}}\,dx}$
\begin{indicacion}
Introducir la expresión \verb|(6x+5)/((x^2+x+1)^2)| en la línea de edición y proceder de la forma indicada en el apartado anterior.
\end{indicacion}
\end{enumerate}


\item Calcular las siguientes integrales:
\begin{enumerate}
\item $ \dint^{0}_{-\frac{1}{2}}{\dfrac{x^{3}}{x^{2}+x+1}}\,dx$
\begin{indicacion}
Introducir \verb|x^3/(x^2+x+1)| en la línea de edición. 
Utilizar el menú \menu{Cálculo > Integrales} y elegir \opcion{Integral Definida}.
Introducir \verb|-1/2| en \opcion{Límite Inferior} y \verb|0| en \opcion{Límite Superior}, y pinchar en \boton{Simplificar}.
\end{indicacion}

\item $ \dint^{4}_{2}{\dfrac{\sqrt{16-x^{2}}}{x}\,dx}$
\begin{indicacion}
Introducir la expresión \verb|sqrt(16-x^2)/x| en la línea de edición y proceder de la forma indicada en el apartado anterior,
introduciendo los valores \verb|2| y \verb|4| en \opcion{Límite Inferior} y \opcion{Límite Superior} respectivamente.
\end{indicacion}

\item $ \dint^{\frac{\pi}{2}}_{0}{\dfrac{dx}{3+\cos(2x)}}$
\begin{indicacion}
Introducir la expresión \verb|1/(3+cos(2x))| y proceder de la forma indicada en el apartado anterior, introduciendo los valores
\verb|0| y \verb|pi/2| en \opcion{Límite Inferior} y \opcion{Límite Superior} respectivamente.
\end{indicacion}
\end{enumerate}


\item Calcular la siguiente integral
\[
\  \dint_{2}^{\infty}{x^{2}e^{-x}\,dx}.
\]
\begin{indicacion}
Introducir la expresión \verb|x^2exp(-x)| en la línea de edición y seguir las indicaciones del el primer apartado del ejercicio anterior,
introduciendo los valores \verb|2| e \verb|inf| en \opcion{Límite Inferior} y \opcion{Límite Superior} respectivamente.
\end{indicacion}


\item Representar la parábola $y=x^{2}-7x+6$, y calcular el área limitada por dicha parábola, el eje de abscisas y las rectas $x=2$ y $x=6$.
\begin{indicacion}
\begin{enumerate}
\item Introducir la expresión \verb|x^2-7x+6| en la línea de edición, se pincha en el botón \boton{Ventana 2D} de la barra de botones
para acceder al entorno de gráficos de dos dimensiones, y una vez allí se pincha en el botón \boton{Representar Expresión}.
\item Para volver a la Ventana de Álgebra se pincha en el botón \boton{Activar la Ventana de Álgebra}, y una vez en ella se realiza lo
indicado en el apartado anterior sucesivamente con las expresiones \verb|x=2| y \verb|x=6| para obtener sus representaciones gráficas.
\item Puede observarse en la gráfica que, entre $x=2$ y $x=6$, la parábola $y=x^{2}-7x+6$ se encuentra por debajo del eje de abscisas, por
lo que si se calcula el valor de la integral definida de $x^{2}-7x+6$ entre esos límites el resultado será negativo. Para hallar el área
encerrada habrá que cambiar el signo al resultado.
\item Seleccionar la expresión de la parábola, el menú \menu{Cálculo > Integrales} y elegir \opcion{Integral Definida}. Introducir
\verb|2| en \opcion{Límite Inferior} y \verb|6| en \opcion{Límite Superior}, y pinchar en \boton{Simplificar}. El área buscada será el
número obtenido cambiado de signo.
\item También podría haberse hallado el área pedida calculando la integral definida de $-(x^{2}-7x+6)$, o la de $|x^{2}-7x+6|$, entre $x=2$
y $x=6$.
\end{enumerate}
\end{indicacion}


\item  Representar gráficamente la región del primer cuadrante limitada por la parábola $y^{2}=8x$, la recta $x=2$ y el eje $OX$, y hallar
el volumen generado en la rotación alrededor del eje $OX$ de la región anterior.
\begin{indicacion}
\begin{enumerate}
\item Se introduce \verb|y^2=8x| en la línea de edición, se pincha en el botón \boton{Ventana 2D} de la barra de botones para acceder
al entorno de gráficos de dos dimensiones, y una vez allí se pincha en el botón \boton{Representar Expresión}.
\item Para volver a la Ventana de Álgebra se pincha en el botón \boton{Activar la Ventana de Álgebra}, y una vez en ella se realiza lo
indicado en el apartado anterior con la expresión \verb|x=2| y se observa en la gráfica la región del primer cuadrante limitada por la
parábola $y^{2}=8x$, la recta $x=2$ y el eje $OX$.
\item Para calcular el volumen generado en la rotación alrededor del eje $OX$ de la región anterior, se calcula la integral de $\pi8x$ entre
$0$ y $2$, siguiendo las indicaciones incluidas en el primer apartado del segundo ejercicio.
\end{enumerate}
\end{indicacion}

\end{enumerate}


\section{Ejercicios propuestos}
\begin{enumerate}[leftmargin=*]
\item Calcular las siguientes integrales:
\begin{enumerate}
\item $ \int{\dfrac{2x^{3}+2x^{2}+16}{x(x^{2}+4)^{2}}\;dx}$
\item $ \int{\dfrac{1}{x^{2}\sqrt{4+x^{2}}}\;dx}$
\end{enumerate}

\item Hallar el área encerrada la parábola $y=9-x^{2}$ y la recta $y=-x$.

\item Hallar el área encerrada por la curva $y=e^{-|x|}$ y su asíntota.

\item Hallar el volumen generado en la rotación alrededor del eje $OX$ de la región plana limitada por la parábola $y=2x^{2}$, las rectas
$x=0$, $x=5$ y el eje $OX$, representando previamente dicha región plana.

\item Hallar el volumen generado en la rotación alrededor del eje $OY$ del área limitada por la parábola $y^{2}=8x$ y la recta $x=2$.

\end{enumerate}
